 %% Private
    Hence, we can summarize this paper as follows. The effects of liquidity constraints and risks are similar because both stem from the same source: a concavification of the consumption function. The effects work independently, meaning that neither risks nor constraints are necessary to concavify the consumption function. And since a more concave consumption function exhibits heightened prudence, both constraints and risks strengthen the precautionary saving motive. In addition, we explain the apparently contradictory results that constraints and risks in some cases intensify, but in other cases weaken the precautionary saving motive. The central insight is that the effect of introducing an additional constraint or risk depends on whether it weakens the effects of any pre-existing constraints or risks. If it does not interact with any pre-existing constraints or risks, it intensifies the precautionary saving motive. If it does interact, it may weaken the precautionary saving motive at some levels of wealth.
  
