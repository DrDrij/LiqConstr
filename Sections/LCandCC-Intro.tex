 %% Private
  This section shows under which conditions liquidity constraints cause consumption concavity. The main conceptual difficulty with liquidity constraints is that the effect of introducing a new constraint depends on already existing constraints. To get around this issue, we introduce the concept of an \textit{ordered} set of \textit{relevant} constraints. This allows us to add constraints in such a way that the next constraint does not affect behavior related to pre-existing constraints. Our main result (Theorem \ref{thm:lcip2}) is that the introduction of the next constraint from the ordered set of relevant constraints causes a counterclockwise concavification of the consumption function. It then follows from the results in Section \ref{sec:PrudAndCC} that the introduction of the next constraint also heightens prudence of the value function.
