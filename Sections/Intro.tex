 %

  Numerical solutions have now supplanted analytical methods for modeling consumption/saving choices, because analytical solutions are not available for realistic descriptions of utility, uncertainty, and constraints.

  A large literature in both micro and macroeconomics has demonstrated that numerical models that take constraints and uncertainty seriously can yield quite different conclusions than those obtainable for traditional models. For example, in heterogeneous agent New Keynesian models (e.g. \citealp{kmvHANK}), a major transmission mechanism for monetary policy is the indirect income effect because a substantial share of households have high marginal propensities to consume -- a channel that is of minimal importance in perfect foresight unconstrained models. And \citet{glLiq} and \citet{blPrecautionary} show that tightened borrowing capacity and heightened income risk may be important explanatory factors behind the consumption decline during the great recession. Further, \citet{kmpHandbook} show that numerically realistic models can match the empirical finding that the drop in consumption spending during the great recession was heavily concentrated in the middle class.

But a drawback to numerical solutions is that it is often difficult to know why results come out the way they do.  A leading example is in the complex relationship between precautionary saving behavior and liquidity constraints.  At least since \citet{zeldes:thesis}, economists working with numerical solutions have known that liquidity constraints can strictly increase precautionary saving under very general circumstances - even for consumers with a quadratic utility function that generates no intrinsic precautionary saving motive.\footnote{For the seminal numerical examination of some of the interactions between precautionary saving and liquidity constraints, see \citet{deatonLiqConstr}.} On the other hand, simulation results have often found circumstances under which liquidity constraints and precautionary saving are substitutes rather than complements.  In an early example, \citet{samwick:pensions} showed that unconstrained consumers with a precautionary saving motive in a retirement saving model behave in ways qualitatively and quantitatively similar to the behavior of liquidity constrained consumers facing no uncertainty.

This paper provides the theoretical tools to make sense of the interactions between liquidity constraints and precautionary saving. These tools provide a rigorous theoretical foundation that can be used to clarify the reasons for the numerical literature's apparently contrasting findings.

For example, one of the paper's main results is a proof that when a liquidity constraint is added to a standard consumption problem, the resulting value function exhibits increased `prudence' (a greater precautionary motive) around the level of wealth where the constraint becomes binding.\footnote{\citet{kimball:smallandlarge} defines prudence of the value function and shows that it is the key theoretical requirement to produce precautionary saving.} Constraints induce precaution because constrained agents have less flexibility in responding to shocks when the effects of the shocks cannot be spread out over time. We show that the precautionary motive is heightened by the desire (in the face of risk) to make future constraints less likely to bind.\footnote{To be clear, the liquidity constraint we analyze here must be satisfied in each period (one-period bonds). This implies that the interactions between constraints and income volatility where some households may prefer to increase (credit card) debt today because they expect tighter credit conditions in the future are ruled out \citep{fulford2015important,druedahl2018precautionary}.}

At a deeper level, we show that the effect of a constraint on prudence is an example of a general theoretical result: Prudence is induced by concavity of the consumption function. Since a constraint creates consumption concavity around the point where the constraint binds,\footnote{Since the first version of this paper, the connection between constraints and consumption concavity has been explored in more specific settings (see e.g. \citet{park2006analytical} for CRRA utility,  \citet{seater1997optimal} for the case where time-discounting equals the interest rate, \citet{nishiyama2012concavity} for quadratic utility, and \citet{holm2018consumption} for the case with infinitely-lived households with HARA utility). The main contribution of the current paper is to derive the implications of liquidity constraints on consumption concavity in a life-cycle setting and to derive the implications of additional constraints at specific points in time on consumption concavity.} adding a constraint necessarily boosts prudence around that point.\footnote{A relationship between constraints and prudence has also been noted by \citet{lee2007degree} and documented empirically in \citet{lee2010precautionary}.} We show that this concavity-boosts-prudence result holds for any utility function with non-negative third derivative; ``prudence'' in the \textit{utility} function as in  \citet{kimball:smallandlarge} is not necessary, because prudence is created by consumption concavity.

These results connect closely to \citet{carroll&kimball:concavity}'s demonstration that, within the HARA utility class, the introduction of uncertainty causes the consumption function to become strictly concave (in the absence of constraints) for all but a few knife-edge combinations of utility function and structure of risk.  Taken together, this paper and \citet{carroll&kimball:concavity} can be seen as establishing rigorously the sense in which precautionary saving and liquidity  constraints are substitutes.\footnote{See \citet{fernandez-corugedo:softlc} for a related demonstration that `soft' liquidity constraints bear an even closer resemblance to precautionary behavior. \citet{MendelsonAmihud:consumption} provide an impressive treatment of a similar problem.} To illustrate this point, we provide an example of a specific kind of uncertainty that (under CRRA utility, in the limit) induces a consumption function that is point-wise identical to the consumption function that would be induced by the addition of a liquidity constraint.

We further show that, once consumption concavity is created (by the introduction of either uncertainty or a constraint, or in any other way), it propagates back to periods before the period in which the concavity has been introduced.\footnote{\citet{carroll&kimball:concavity} showed that the concavity induced by uncertainty propagated backwards, but the proofs in that paper cannot be applied to concavity created by a liquidity constraint.} Precautionary saving is induced by the \textit{possibility} that constraints might bind; this can explain why such a high percentage of households cite precautionary motives as the most important reason for saving \citep{kennickell&lusardi:scfquestions} even though the fraction of households who report actually having been constrained in the past is relatively low \citep{jappelli:whoisconstr}.

Our final theoretical contribution is to show that the introduction of further liquidity constraints beyond the first one may actually \textit{reduce} precautionary saving at some levels of wealth by `hiding' the effects of the pre-existing constraint(s); they are no longer relevant because the liquidity constraint forces more saving than the precautionary motive would induce. Identical logic implies that uncertainty can `hide' the effects of a constraint, because the consumer may save so much for precautionary reasons that the constraint becomes irrelevant. For example, a typical perfect foresight model of retirement consumption for a consumer with Social Security (guaranteed pension) income implies that a legal constraint on borrowing against benefits will cause the consumer to run assets down to zero, and thereafter set consumption equal to income.  Now consider adding the possibility of large medical expenses near the end of life (e.g.\ nursing home fees; see \citealp{aclvJoy}).  Under reasonable assumptions, a consumer facing such a risk may save enough for precautionary reasons to render the no-borrowing constraint irrelevant.

Our analysis proceeds in five steps. We present our general theoretical framework in the next section. We then show that consumption concavity increases prudence (Section \ref{sec:PrudAndCC}); that concavity, once created, propagates to previous periods (Section \ref{sec:Aggregation}); that constraints cause consumption concavity (Section \ref{sec:Precaution}); and when additional constraints or risks increase the precautionary saving motive (Section \ref{sec:ConstrRisksCPPandPS}). The final section concludes.

%There are three steps: consumption concavity increases prudence (), liquidity constraints cause consumption concavity
%
%
%The rest of the paper is structured as follows. To fix notation and ideas, the next section sets out our general theoretical framework. The third section shows that concavity of the consumption function heightens prudence.  The fourth section shows how concavity, whether induced by constraints or uncertainty, propagates to previous periods. Section 5 shows how the introduction of a constraint creates a precautionary saving motive. The sixth section shows when the introduction of additional liquidity constraints beyond the first constraint increases the precautionary motive at any given level of wealth.  The fact that this does not always occur means that the introduction of constraints or uncertainty can weaken the effects of pre-existing constraints or uncertainty. The final section concludes.

