 %% Private




  \section{Proof of Lemma \ref{lem:recursive}}\label{app:recursive}

  \begin{proof}
    First, to facilitate readability of the proof, we assume that $R = \beta = 1$ with no loss of generality. Our goal is to prove that $V({m}_{t}) \in CC$ if $V_{t+1}({a}_{t} + {y}_{t+1}) \in CC$ for all realizations of ${y}_{t+1}$. The proof proceeds in two steps. First, we show that property CC is preserved through the expectation operator (vertical aggregation), i.e.\ that $\Omega({a}_{t}) = \Ex_t[V_{t+1}({a}_{t} + {y}_{t+1})] \in CC$ if $V_{t+1}({a}_{t} + {y}_{t+1}) \in CC$ for all realizations of ${y}_{t+1}$. Second, we show that property CC is preserved through the value function operator (horizontal aggregation), i.e.\ that $V({m}_{t}) = \max_{s} u(c_t({m}_{t} - s)) + \Omega(s) \in CC$ if $\Omega(s) \in CC$. Throughout the proof, the first order condition holds with equality since no liquidity constraint applies at the end of period $t$.

    \bigskip
    \noindent \textbf{Step 1: Vertical aggregation} \\
    \noindent We show that consumption concavity is preserved under vertical aggregation for three cases of the HARA utility function with $u''' \geq 0$ ($\alpha_1  \geq -1$) and non-increasing absolute prudence ($\alpha_1  \notin (-1,0)$). The three cases are
    \begin{equation}\label{eq:HARAmu}u'(c) = \begin{cases} \left(\alpha_1 c + b\right)^{-1/\alpha_1 } & \alpha_1  \in (0,\infty) \text{ (CRRA)} \\
	e^{-c/b} & \alpha_1  = 0 \text{ (CARA)}\\
	\alpha_1 c + b & \alpha_1  = -1 \text{ (Quadratic)}\end{cases} \end{equation}

    \bigskip
    \noindent \textbf{Case I ($\alpha_1  > 0$, CRRA.)} 	We will show that concavity is preserved under vertical aggregation for $c^{-1/\alpha_1 }$ to avoid clutter, but the results hold for all affine transformations, $\alpha_1 c + b$, with $\alpha_1 >0$. Concavity of $c_{t+1}({a}_{t} + {y}_{t+1})$ implies that
    \begin{equation}c_{t+1}({a}_{t} + {y}_{t+1}) \geq pc_{t+1}({a}_1 + {y}_{t+1}) + (1-p) c_{t+1}({a}_2 + {y}_{t+1}) \label{eq:vert_crra_conc}\end{equation}
    for all ${y}_{t+1} \in [\underline{y},\bar{y}]$ if ${a}_{t} = p{a}_1 + (1-p){a}_2$ with $p \in [0,1]$. Since this holds for all ${y}_{t+1}$, we know that
    \[\left\{\Ex_t\left[c_{t+1}({a}_{t} + {y}_{t+1})^{-\frac{1}{a}}\right]\right\}^{-a} \geq \left\{\Ex_t\left[\left\{pc_{t+1}({a}_1 + {y}_{t+1}) + (1-p) c_{t+1}({a}_2 + {y}_{t+1})\right\}^{-\frac{1}{\alpha_1}}\right]\right\}^{-\alpha_1}\]
    We now apply Minkowski's inequality (see e.g.\ \citealp[Theorem 3]{beckenbach1983inequalities}) which says that for $u,v \geq 0$ and a scalar $k < 1\ (k \neq 0)$
    \[\left\{\Ex[(u + v)^{k}]\right\}^{1/k} \geq \left\{\Ex[u^{k}]\right\}^{1/k} + \left\{\Ex[v^{k}]\right\}^{1/k}.\]
    %	
    %	
    %	\[\left\{\Ex[(u + v)^{-k}]\right\}^{-1/k} \geq \left\{\Ex[u^{-k}]\right\}^{-1/k} + \left\{\Ex[v^{-k}]\right\}^{-1/k}.\]
    This implies that for $\alpha_1  \in (0,\infty)$ (CRRA)
    \[\left\{\Ex[(u + v)^{-\frac{1}{\alpha_1 }}]\right\}^{-\alpha_1 } \geq \left\{\Ex[u^{-\frac{1}{\alpha_1 }}]\right\}^{-\alpha_1 } + \left\{\Ex[v^{-\frac{1}{\alpha_1 }}]\right\}^{-\alpha_1 }\]
    if $u\geq 0$ and $v\geq 0$. Thus
    \begin{align*}
      &\left\{ \Ex_t \left[\left\{pc_{t+1}({a}_1 + {y}_{t+1}) + (1-p) c_{t+1}({a}_2 + {y}_{t+1})\right\}^{-\frac{1}{\alpha_1 }}\right]\right\}^{-\alpha_1 } \\ & \geq \left\{ \Ex_t \left[\left\{pc_{t+1}({a}_1 + {y}_{t+1})\right\}^{-\frac{1}{\alpha_1 }}\right]\right\}^{-\alpha_1 } + \left\{ \Ex_t \left[\left\{(1-p) c_{t+1}({a}_2 + {y}_{t+1})\right\}^{-\frac{1}{\alpha_1 }}\right]\right\}^{-\alpha_1 } \\
      & = p\left\{ \Ex_t \left[\left\{c_{t+1}({a}_1 + {y}_{t+1})\right\}^{-\frac{1}{\alpha_1 }}\right]\right\}^{-\alpha_1 } +(1-p) \left\{ \Ex_t \left[\left\{ c_{t+1}({a}_2 + {y}_{t+1})\right\}^{-\frac{1}{\alpha_1 }}\right]\right\}^{-\alpha_1 } \\ & = p(\Omega'({a}_1))^{-\alpha_1 } + (1-p)(\Omega'({a}_2))^{-\alpha_1 }
    \end{align*}
    which implies that
    \[ (\Omega'({a}_{t}))^{-\alpha_1 } \geq p(\Omega'({a}_1))^{-\alpha_1 } + (1-p)(\Omega'({a}_2))^{-\alpha_1 }\]	
    Thus, defining
    $\chi_{t}({a}_{t}) = \{\Omega_{t}^{'}({a}_{t})\}^{-\alpha_1 }$, we get
    \begin{align*}
      \chi_{t}({a}_{t}) \geq p \chi_{t}({a}_{1}) + (1-p) \chi_{t}({a}_{2})
    \end{align*}
    for all ${a}_{t}$, where the inequality is strict if $c_{t+1}$ is strictly concave for at least one realization of ${y}_{t+1}$.

    \bigskip
    \noindent \textbf{Case II ($a = 0$, CARA)}. For the exponential case, property CC holds at ${a}_{t}$ if
    \begin{eqnarray*}
      \exp(-\chi_{t}({a}_{t})/b) & = & \Ex_{t}[ \exp(-c_{t+1}({a}_{t}+{y}_{t+1})/b)]
    \end{eqnarray*}
    for some $\chi_{t}({a}_{t})$ which is strictly concave at ${a}_{t}$. We set $b = 1$ to reduce clutter, but results hold for $b \neq 1$. Consider first a case where $c_{t+1}$ is linear over the range of possible values of ${a}_{t}+{y}_{t+1}$, then
    \begin{eqnarray}
      \chi_{t}({a}_{t}) & = & -\log \Ex_{t}[e^{-c_{t+1}({a}_{t}+{y}_{t+1})}] \nonumber
      \\& = & -\log \Ex_{t}[e^{-(c_{t+1}({a}_{t}+\bar{y})+({y}_{t+1}-\bar{y})c_{t+1}^{'})}] \nonumber
      \\ & = & c_{t+1}({a}_{t}+\bar{y}) - \log \Ex_{t}[e^{-({y}_{t+1}-\bar{y}) c_{t+1}^{'}}] \label{eq:expcdrops}
    \end{eqnarray}
    which is linear in ${a}_{t}$ since the second term is a constant.

    Now consider a value of ${a}_{t}$ for which $c_{t+1}({a}_{t}+{y}_{t+1})$ is strictly concave for at least one realization of ${y}_{t+1}$. Global weak concavity of $c_{t+1}$ tells us that for every ${y}_{t+1}$
    \begin{eqnarray}
      -c_{t+1}({a}_{t}+{y}_{t+1}) & \leq & -((1-p)c_{t+1}({a}_1+{y}_{t+1})+pc_{t+1}({a}_{2}+{y}_{t+1})) \nonumber
      \\ \Ex_{t}[e^{-c_{t+1}({a}_{t}+{y}_{t+1})}] & \leq & \Ex_{t}[e^{-((1-p)c_{t+1}({a}_1+{y}_{t+1})+pc_{t+1}({a}_{2}+{y}_{t+1}))}]. \label{eq:expconc}
    \end{eqnarray}

    Meanwhile, the arithmetic-geometric mean inequality states that for positive $u$ and $v$, if $\bar u = \Ex_t [ u]$ and $\bar v = \Ex_t [ v]$, then
    \begin{equation*}
      \Ex_t\left[( u/\bar u)^p ( v/\bar v)^{1-p}\right]
      \leq \Ex_t \left[p ( u/\bar u)+(1-p)( v/\bar v)\right] = 1,
    \end{equation*}
    implying that
    \begin{equation*}
      \Ex_t [ u^p  v^{1-p}] \leq \bar u^p \bar v^{1-p},
    \end{equation*}
    where the expression holds with equality only if $v$ is proportional to $u$.  Substituting in $u= e^{-c_{t+1}({a}_1 + {y}_{t+1})}$ and $v = e^{-c_{t+1}({a}_2 + {y}_{t+1})}$, this means that
    \begin{eqnarray*}
      \Ex_t [e^{-p c_{t+1}({a}_1 + {y}_{t+1}) - (1-p)c_{t+1}({a}_2 + {y}_{t+1})}] & \leq &
                                                                                           \left\{ \Ex_t  [e^{- c_{t+1}({a}_1 + {y}_{t+1})} ] \right\}^p \left\{ \Ex_t  [e^{- c_{t+1}({a}_2 + {y}_{t+1})} ] \right\}^{1-p}
    \end{eqnarray*}
    and we can substitute for the LHS from \eqref{eq:expconc}, obtaining
    \begin{eqnarray}
      \Ex_{t}[e^{-c_{t+1}({a}_{t}+{y}_{t+1})}] & \leq &
                                                        \left\{ \Ex_t  [e^{- c_{t+1}({a}_1 + {y}_{t+1})} ] \right\}^p 	\left\{ \Ex_t  [e^{- c_{t+1}({a}_2 + {y}_{t+1})} ] \right\}^{1-p} \nonumber
      \\
      \log \Ex_{t}[e^{-c_{t+1}({a}_{t}+{y}_{t+1})}] &\leq &
                                                            p \log \Ex_t  [e^{-c_{t+1}({a}_1 + {y}_{t+1})} ]
                                                            + (1-p) \log \Ex_t  [e^{-c_{t+1}({a}_2 + {y}_{t+1})} ]  \label{eq:expineq2}
    \end{eqnarray}
    which holds with equality only when $e^{-c_{t+1}({a}_{1}+{y}_{t+1})}/e^{-c_{t+1}({a}_{2}+{y}_{t+1})}$ is a constant. This will only happen if 	$c_{t+1}({a}_{1}+{y}_{t+1})-c_{t+1}({a}_{2}+{y}_{t+1})$ is constant, which (given that the MPC is strictly positive everywhere) requires 	$c_{t+1}({a}_{t}+{y}_{t+1})$ to be linear for ${y}_{t+1} \in (\underline{y},\bar{y})$. Hence,
    \begin{eqnarray*}
      \chi_{t}({a}_{t}) & \geq & p \chi_{t}({a}_{1}) + (1-p) \chi_{t}({a}_{2}).
    \end{eqnarray*}
    where the inequality is strict for an ${a}_{t}$ from which $c_{t+1}$ is strictly concave for some realization of ${y}_{t+1}$.

    \bigskip
    \noindent \textbf{Case III ($a = -1$, Quadratic)}.
    In the quadratic case, linearity of marginal utility implies that
    \begin{eqnarray*}
      u'(\chi_{t}({a}_{t})) & = & \Ex_{t}[u'(c_{t+1}({a}_{t}+{y}_{t+1}))]
      \\   \chi_{t}({a}_{t}) & = & \Ex_{t}[c_{t+1}({a}_{t}+{y}_{t+1})]
    \end{eqnarray*}
    so $\chi_{t}$ is simply the weighted sum of a set of
    concave functions where the weights correspond to the probabilities of the various possible outcomes for ${y}_{t+1}$. The sum of concave functions is itself concave. And if additionally the consumption function is strictly concave at any point, the weighted sum is also strictly concave.


    \bigskip
    \noindent \textbf{Step 2: Horizontal aggregation:} \\
    We now proceed with horizontal aggregation, namely how concavity is preserved through the value function operation. Assume that $\Omega_t({a}_{t}) \in CC$ at point ${a}_{t}$, then the first order condition implies that
    \[\Omega_t'({a}_{t}) = u'(\chi_t({a}_{t}))\]
    for some monotonically increasing $\chi_t({a}_{t})$ that satisfies
    \begin{align}
      \chi_t(p{a}_1 + (1-p){a}_2) \geq p\chi_t({a}_1) + (1-p)\chi_t({a}_2) \label{eq:hor_conc}
    \end{align}
    for any $0 < p < 1$, and ${a}_1 < {a}_{t} < {a}_2$. % ($|{a}_2 - {a}_1|< \delta$ for any $\delta > 0$).

    In addition, we know that the first order condition holds with equality such that $\Omega_t'({a}_{t}) = u'(c_t({m}_{t})) = u'(\chi_t({a}_{t}))$ which implies that ${a}_{t} = \chi_t^{-1}(c_t)$. Using this equation, we get
    \begin{align*}
      \chi_t(p{a}_1 + (1-p){a}_2) &\geq p\chi_t({a}_1) + (1-p)\chi_t({a}_2) \\
      p{a}_1 + (1-p){a}_2 &\geq \chi_t^{-1}(p\chi_t({a}_1) + (1-p)\chi_t({a}_2)) \\
      p\chi_t^{-1}(c_1) + (1-p)\chi_t^{-1}(c_2) &\geq \chi_t^{-1}(pc_1 + (1-p)c_2)
    \end{align*}
    which implies that $\chi_t^{-1}$ is a convex function.

    Use the budget constraint to define
    \begin{align*}
      {m}_{t} &= {a}_{t} + c_t \\
      \wAlt(c_t) & = \chi^{-1}(c_t) + c_t
    \end{align*}
    Now, since $\chi_t^{-1}$ is a convex function, and $\wAlt(c_t)$ is the sum of a convex and a linear function, it is also a convex function satisfying
    \begin{align}
      p\wAlt(c_1) + (1-p)\wAlt(c_2) &\geq \wAlt(pc_1 + (1-p)c_2) \nonumber \\
      \wAlt^{-1}(p\wAlt(c_1) + (1-p)\wAlt(c_2)) &\geq pc_1 + (1-p)c_2 \nonumber \\
      c(p{m}_1 + (1-p){m}_2) &\geq pc({m}_1) + (1-p)c({m}_2) \label{eq:hor_conc2}
    \end{align}
    so $c$ is concave.

    Note that the proof of horizontal aggregation works for any utility function with $u' > 0$ and $u'' < 0$ when $R$ = $\beta$ = 1. However, for the more general case where $R$ or $\beta$ are not equal to one, we need the HARA property that multiplying $u'$ by a constant corresponds to a linear transformation of $c$.

    \bigskip
    \noindent \textbf{Strict Consumption Concavity.}
    When $V_{t+1}({m}_{t+1})$ exhibits the property strict CC for at least one ${m}_{t+1} \in [R{a}_{t} + \underline{y}, R{a}_{t} + \bar{y}]$, we know that $\chi_t({a}_{t})$ also exhibits the property strict CC from the proof of vertical aggregation. Then, equation \eqref{eq:hor_conc} holds with strict inequality, and this strict inequality goes through the proof of horizontal aggregation, implying that equation \eqref{eq:hor_conc2} holds with strict inequality. Hence, $c_t({m}_{t})$ is strictly concave if $c_{t+1}({a}_{t} + {y}_{t+1})$ is concave for all realizations of ${y}_{t+1}$ and strictly concave for at least one realization of ${y}_{t+1}$.
  \end{proof}


  \section{Proof of Lemma \ref{lem:counterclockwise}} \label{app:counterclockwise}
  \begin{proof} First, condition 2 and 4 in Definition \ref{defn:cconcavification} imply that $\hat{c}'({m}) > c'({m})$ for ${m} = {m}^{\#} - \epsilon$ for a small $\epsilon > 0$. Condition 3 then ensures that $\lim_{\mu \uparrow {m}} \hat{c}'(\mu) > \lim_{\mu \uparrow {m}} c'(\mu)$ holds for all ${m} \leq {m}^{\#}-\epsilon$ (equivalently ${m} < {m}^{\#}$). Second, condition 1 and the fact that $\lim_{\mu \uparrow {m}} \hat{c}'(\mu) > \lim_{\mu \uparrow {m}} c'(\mu)$ for ${m} < {m}^{\#}$ implies that $\lim_{\mu \uparrow {m}} \hat{c}(\mu) < \lim_{\mu \uparrow {m}}c(\mu)$ for ${m} < {m}^{\#}$. Third, condition 3 in Definition \ref{defn:cconcavification} implies that $$\lim_{\mu \uparrow {m}}\hat{c}''(\mu) \leq \lim_{\mu \uparrow {m}} c''(\mu)\frac{\hat{c}'(\mu)}{c'(\mu)}$$ for ${m} < {m}^{\#}$. Then $$\lim_{\mu \uparrow {m}} \hat{c}''(\mu) \leq \lim_{\mu \uparrow {m}} c''(\mu)$$ since $\lim_{\mu \uparrow {m}}\hat{c}'(\mu) > \lim_{\mu \uparrow {m}} c'(\mu)$ for ${m} < {m}^{\#}$. Note that the inequality is not strict since $c''(\mu)$ could be 0.
  \end{proof}


  \section{Proof of Lemma \ref{lem:CCToPrud}} \label{app:CCToPrud}
  \begin{proof}
    By the envelope theorem, we know that
    \[ V'({m}) = u'(c({m}))\]
    Differentiating with respect to ${m}$ yields
    \begin{equation}\label{eq:diff2}
      V''({m}) = u''(c({m}))c'({m})
    \end{equation}
    Since $c({m})$ is concave, it has left-hand and right-hand derivatives at every point, though the left-hand and right-hand derivatives may not be equal. Equation \eqref{eq:diff2} should be interpreted as applying the left-hand and right-hand derivatives separately. (Reading \eqref{eq:diff2} in this way implies that $c'({m}^-) \geq c'({m}^+)$; therefore $V''({m}^-) \leq V''({m}^+)$). Taking another derivative can run afoul of the possible discontinuity in $c'({m})$ that we will show below can arise from liquidity constraints. We therefore consider two cases: (i) $c''({m})$ exists and (ii) $c''({m})$ does not exist.

    \bigskip
    \noindent \textbf{\textit{Case I:}} ($c''({m})$ exists.)\\
    In the case where $c''({m})$ exists, we can take another derivative
    \[V'''({m}) = u'''(c({m}))[c'({m})]^2 + u''(c({m}))c''({m})\]
    Absolute prudence of the value function is thus defined as
    \begin{align}-\frac{V'''({m})}{V''({m})} &= -\frac{u'''(c({m}))[c'({m})]^2 + u''(c({m}))c''({m})}{u''(c({m}))c'({m})} \nonumber \\
      -\frac{V'''({m})}{V''({m})} &= -\frac{u'''(c({m}))}{u''(c({m}))}c'({m}) - \frac{c''({m})}{c'({m})}\label{eq:absprudence}\end{align}
    From the assumption that $\hat{c}({m})$ is a counterclockwise concavification of $c({m})$, we know from Lemma \ref{lem:counterclockwise} that  $\hat{c}({m}) \leq c({m})$ and $\hat{c}'({m}) \geq c'({m})$. Furthermore, since $-\frac{u'''(c({m}))}{u''(c({m}))}$ is non-increasing, we know that $-\frac{u'''(\hat{c}({m}))}{u''(\hat{c}({m}))} \geq -\frac{u'''(c({m}))}{u''(c({m}))}$. As a result, $-\frac{u'''(\hat{c}({m}))}{u''(\hat{c}({m}))}\hat{c}'({m}) \geq -\frac{u'''(c({m}))}{u''(c({m}))}c'({m})$.

    The second part of the absolute prudence expression, $-\frac{c''({m})}{c'({m})}$, is a measure of the curvature of the consumption function. Since the consumption function is concave, $-\frac{c''({m})}{c'({m})}$ is a measure of the degree of concavity. Formally, if one has two functions, $f(x)$ and $g(x)$, that are both increasing and concave functions, then the concave transformation $g(f(x))$ always has more curvature than $f$.\footnote{To see this, compute \[-\frac{\frac{d^2}{dx^2} g(f(x))}{\frac{d}{dx}g(f(x))} = - \frac{g''f'}{g'} - \frac{f''}{f'} \geq - \frac{f''}{f'}\] where the inequality holds since $f' \geq 0$, $g' \geq 0$, and $g'' \leq 0$.} A counterclockwise concavification is an example of such a $g$. Hence, $-\frac{\hat{c}''({m})}{\hat{c}'({m})} \geq -\frac{c''({m})}{c'({m})}$. Then
    \begin{align*}
      -\frac{\hat{V}'''({m})}{\hat{V}''({m})} &= -\frac{u'''(\hat{c}({m}))}{u''(\hat{c}({m}))}\hat{c}'({m}) - \frac{\hat{c}''({m})}{\hat{c}'({m})} \\
                                              &\geq -\frac{u'''(c({m}))}{u''(c({m}))}c'({m}) - \frac{c''({m})}{c'({m})} = -\frac{V'''({m})}{V''({m})}
    \end{align*}

    \bigskip
    \noindent \textbf{\textit{Case II:}} ($c''({m})$ does not exist.)\\
    Informally, if nonexistence is caused by a constraint binding at ${m}$, the effect will be a discrete decline in the marginal propensity to consume at ${m}$, which can be thought of as $c''({m}) = -\infty$, implying positive infinite prudence at that point (see \eqref{eq:absprudence}). Formally, if $c''({m})$ does not exist, greater prudence of $\hat{V}$ than $V$ is given by $\frac{\hat{V}''({m})}{V''({m})}$ being a decreasing function of ${m}$. This is defined as
    \[\frac{\hat{V}''({m})}{V''({m})} \equiv
      \left(\frac{u''(\hat c({m}))}{u''(c({m}))} \right)
      \left(\frac{\hat{c}'({m})}{c'({m})}\right)\]
    The second factor, $\frac{\hat{c}'({m})}{c'({m})}$, is weakly decreasing in ${m}$ by the property of a counterclockwise concavification. At any specific value of ${m}$ where
    $\hat{c}''({m})$ does not exist because the left and right hand
    values of $\hat{c}'$ are different, we say that $\hat{c}'$
    is decreasing if
    \begin{eqnarray}
      \lim_{{m}^{-} \rightarrow {m}} \hat{c}'({m}) & > & \lim_{{m}^{+} \rightarrow {m}} \hat{c}'({m}).
    \end{eqnarray}

    As for the first factor, note that nonexistence of 	$\hat{V}'''({m})$ and/or $\hat{c}''({m})$ do not spring from nonexistence of either $u'''(c)$ or $\lim_{{m} \uparrow {m}} \hat{c}'({m})$ (for our purposes, when the left and right derivatives of $\hat{c}({m})$ differ at a point, the relevant derivative is the one coming from the left; rather than carry around the cumbersome limit notation, read the following derivation as applying to the left derivative).  To discover whether $\frac{\hat {V}''({m})}{V''({m})}$ is decreasing we differentiate $\log\left(\frac{u''(\hat{c}({m}))}{ u''(c({m}))}\right)$ (recall that the log is a monotonically decreasing transformation so the derivative of the log of a function always has the same sign as the derivative of the function):
    \begin{align*}
      \frac{d}{d {m}}\left( \log(u''(\hat{c}({m})) - \log(u''(c({m})))\right) = \frac{u'''(\hat{c}({m}))}{u''(\hat{c}({m}))}\hat{c}'({m}) -  \frac{u'''(c({m}))}{u''(c({m}))}c'({m}).
    \end{align*}
    This will be negative if
    \begin{align}
      \frac{u'''(\hat{c}({m}))}{u''(\hat{c}({m}))}\hat{c}'({m}) & \leq  \frac{u'''(c({m}))}{u''(c({m}))}c'({m}) \nonumber
      \\ \Rightarrow  -\frac{u'''(\hat{c}({m}))}{u''(\hat{c}({m}))}\hat{c}'({m}) & \geq  -\frac{u'''(c({m}))}{u''(c({m}))} c'({m}) \label{eq:prudcond} .
    \end{align}

    Recall from Lemma \ref{lem:counterclockwise} that $\hat{c}'({m}) \geq c'({m})$ and $\hat{c}({m}) \leq c({m})$ so non-increasing absolute prudence of the utility function ensures that $-\frac{u'''(\hat{c}({m}))}{u''(\hat{c}({m}))} \geq  -\frac{u'''(c({m}))}{u''(c({m}))}$. Hence the LHS is always greater or equal to the RHS of equation \eqref{eq:prudcond}.

  \end{proof}



  \section{Proof of Lemma \ref{lem:ccandstrictprud}} \label{app:ccandstrictprud}
  \begin{proof}
    We prove each statement in Lemma \ref{lem:ccandstrictprud} separately.

    \bigskip
    \noindent \textbf{\textit{Case I:}} ($ u ''' > 0$.) \\
    If $u''' > 0$, a counterclockwise concavification around ${m}^{\#}$ implies that $\hat{c}({m}) < c({m})$ and $\hat{c}'({m}) > c'({m})$ for all ${m} < {m}^{\#}$. Then \[-\frac{u'''(\hat{c}({m}))}{u''(\hat{c}({m})}\hat{c}'({m}) > -\frac{u'''({c}({m}))}{u''({c}({m}))}{c}'({m}) \text{ for } {m} < {m}^{\#} \]
    Note that this condition is sufficient to prove Lemma \ref{lem:ccandstrictprud} for the case where $c''({m})$ does not exist since it then satisfies \eqref{eq:prudcond}. In the case where $c''({m})$ does exist, we know that
    \[-\frac{\hat{c}''({m})}{\hat{c}'({m})} \geq -\frac{{c}''({m})}{{c}'({m})} \text{ for } {m} < {m}^{\#}\]
    from the proof of Lemma \ref{lem:CCToPrud}. Hence,
    \begin{align*}- \frac{\hat{V}'''({m})}{\hat{V}''({m})} &= -\frac{u'''(\hat{c}({m}))}{u''(\hat{c}({m})}\hat{c}'({m}) - \frac{\hat{c}''({m})}{\hat{c}'({m})}
      \\
                                                           & > -\frac{u'''({c}({m}))}{u''({c}({m}))}{c}'({m}) - \frac{{c}''({m})}{{c}'({m})} = - \frac{{V}'''({m})}{{V}''({m})} \text{ for } {m} < {m}^{\#}
    \end{align*}
    and Lemma \ref{lem:ccandstrictprud} holds in the case with $u''' > 0$ and ${m} < {m}^{\#}$.

    \bigskip
    \noindent \textbf{\textit{Case II:}} ($u''' = 0$.) \\
    The quadratic case requires a different approach. Note first that the conditions in Lemma \ref{lem:ccandstrictprud} hold only below the bliss point for quadratic utility. In addition, since $u'''(\cdot) = 0$, strict inequality between the prudence of $\hat{V}$ and the prudence of $V$ hold only at those points where $\hat{c}(\cdot)$ is strictly concave.

    Recall from the proof of Lemma \ref{lem:CCToPrud} that greater prudence of $\hat{V}({m})$ than $V({m})$ occurs if  $\frac{\hat{V}''({m})}{V''({m})}$ is decreasing in ${m}$. In the quadratic case
    \begin{equation}
      \frac{\hat{V}''({m})}{V''({m})} = \frac{u''(\hat{c}({m}))}{u''(c({m}))} \frac{\hat{c}'({m})}{c'({m})} = \frac{\hat{c}'({m})}{c'({m})}
    \end{equation}
    where the second equality follows since $u''(\cdot)$ is constant with quadratic utility. Thus, prudence is strictly greater in the modified case only if $ \frac{\hat{c}'({m})}{c'({m})}$ strictly declines in ${m}$.
  \end{proof}



  \section{Proof of Lemma \ref{lem:LcAndCc}}\label{app:pfclc}
  We prove Lemma \ref{lem:LcAndCc} by induction in two steps. First, we show that all results in Lemma \ref{lem:LcAndCc} hold when we add the first constraint. The second step is then to show that the results hold when we go from $n$ to $n+1$ constraints. 	

  \begin{lemma}\textit{$(c_{t}' < c_{t+1}')$} \\
    Consider an agent who has a utility function with $u' > 0$ and $u'' < 0$, faces constant income, is impatient ($\beta R < 1$), and has a finite life. Then $c_{t}' < c_{t+1}'$.
  \end{lemma}

  \begin{proof}
    The marginal propensity to consume in period $t$ can be obtained from the MPC in period $t+1$ from the Euler equation
    \begin{eqnarray*}
      u'(c_{t}({m}_{t})) & =  & \beta R u'(c_{t+1}(R({m}_{t}-c_{t}({m}_{t}))+{y})).
    \end{eqnarray*}
    Differentiating both sides with respect to ${m}_{t}$ and omitting arguments to reduce
    clutter we obtain
    \begin{eqnarray*}
      u^{\prime\prime}(c_{t})c_{t}^{\prime} & =  & \beta R u^{\prime\prime}(c_{t+1})c_{t+1}^{\prime}R(1-c_{t}^{\prime}) \nonumber
      \\ \nonumber (u^{\prime\prime}(c_{t}) + \beta R u^{\prime\prime}(c_{t+1})c_{t+1}^{\prime}R)c_{t}^{\prime} & = & \beta R u^{\prime\prime}(c_{t+1})Rc_{t+1}^{\prime}
      \\ \frac{c_{t+1}^{\prime}}{c_{t}^{\prime}}   & = & \frac{u^{\prime\prime}(c_{t}) + \beta R u^{\prime\prime}(c_{t+1})c_{t+1}^{\prime}R}{\beta R u^{\prime\prime}(c_{t+1})R} \label{eq:mpcPF} \\
      \frac{c_{t+1}^{\prime}}{c_{t}^{\prime}}   & = & \frac{u^{\prime\prime}(c_{t})}{\beta R u^{\prime\prime}(c_{t+1})R} + c_{t+1}^{\prime} %\\ \frac{c_{t+1}^{\prime}}{c_{t}^{\prime}} & > &  1
    \end{eqnarray*}
    Since $\beta R < 1$ ensures that $c_{t} > c_{t+1}$, we know that
    \[\frac{u^{\prime\prime}(c_{t})}{\beta R u^{\prime\prime}(c_{t+1})R} \geq  \frac{u^{\prime\prime}(c_{t+1})}{\beta R u^{\prime\prime}(c_{t+1})R} = \frac{1}{\beta R R} > \frac{1}{R}\]
    Furthermore, we know that
    \[c_{t}^{\prime} \geq \frac{R-1}{R}\]
    since $\frac{R-1}{R}$ is the MPC for an infinitely-lived agent with $\beta R = 1$. Hence,
    \[\frac{c_{t+1}^{\prime}}{c_{t}^{\prime}} = \left(\frac{u^{\prime\prime}(c_{t})}{\beta R u^{\prime\prime}(c_{t+1})R} + c_{t}^{\prime} \right) > \frac{1}{R} + \frac{R-1}{R} = 1\]
    and it follows that $c_{t}^{\prime} <  c_{t+1}^{\prime}$.
  \end{proof}

  \begin{lemma}\label{lem:pfclc}(Consumption with one Liquidity Constraint.) \\
    Consider an agent who has a utility function with $u'> 0 $ and $u'' < 0$, faces constant income, ${y}$, and is impatient, $\beta R < 1$. Assume that the agent faces a set $\mathcal{T}$ of one relevant constraint. Then $c_{t,1}({m})$ is a counterclockwise concavification of $c_{t,0}({m})$ around $\wAlt_{t,1}$.
  \end{lemma}


  \begin{proof}
    We now prove Lemma \ref{lem:pfclc} by first showing that the consumption function including the constraint at the end of period $\tau$ is a counterclockwise concavification of the unconstrained consumption function in period $\tau$. Next, we show how the constraint further implies that the consumption function including the constraint is a counterclockwise concavification of the unconstrained consumption function in periods prior to $\tau$.

    We first define $\tau = \mathcal{T}[1]$ as the time period of the constraint. Note first that consumption is unaffected by the constraint for all periods after $\tau$, i.e. $c_{\tau+k,1}=c_{\tau+k,0}$ for any $k > 0$. For period $\tau$, we can calculate the level of consumption at which the constraint binds by realizing that a consumer for whom the constraint binds will save nothing and therefore arrive in the next period with no wealth. Further, the maximum amount of consumption at which the constraint binds will satisfy the Euler equation (only points where the constraint is strictly binding violate the Euler equation; the point on the cusp does not). Thus, we define $c_{\tau,1}^{\#}$ as the maximum level of consumption in period $\tau$ at which the agent leaves no wealth for the next period, i.e.\ the constraint stops binding:
    \begin{eqnarray*}
      \label{eq:ctau1}
      u'(c_{\tau,1}^{\#})   & = & \beta R u'(c_{\tau+1,0}({y}))
      \\   c_{\tau,1}^{\#}       & = & (u')^{-1}\left(\beta R u'(c_{\tau+1,0}({y}))\right),
    \end{eqnarray*}
    and the level of wealth at which the constraint stops binding can be obtained from
    \begin{eqnarray}
      \label{eq:omegaFromc}
      \wAlt_{\tau,1} & = & \left(V'_{\tau,1}\right)^{-1}(u'(c_{\tau,1}^{\#}))  .
    \end{eqnarray}

    Below this level of wealth, we have $c_{\tau,1}({m}) = {m}$ so the MPC is one, while above it we have $c_{\tau,1}({m}) = c_{\tau,0}({m})$ where the MPC equals the constant MPC for an unconstrained perfect foresight optimization problem with a horizon of $T-\tau$. Thus, $c_{\tau,1}$ satisfies our definition of a counterclockwise concavification of $c_{\tau,0}$ around $\wAlt_{\tau,1}$.

    Further, we can obtain the value of period $\tau-1$ consumption at which the period $\tau$ constraint stops impinging on period $\tau-1$ behavior from
    \begin{eqnarray*}
      u'(c_{\tau-1,1}^{\#})   & = & \beta R u'(c_{\tau,1}^{\#}) \label{eq:ctaum1}
    \end{eqnarray*}
    and we can obtain $\wAlt_{\tau-1,1}$ via the analogue to \eqref{eq:omegaFromc}. Iteration generates the remaining $c_{.,1}^{\#}$ and $\wAlt_{.,1}$ values back to period $t$.

    Now consider the behavior of a consumer in period $\tau-1$ with a level of wealth ${m}<\wAlt_{\tau-1,1}$.  This consumer knows he will be constrained and will spend all of his resources next period, so at ${m}$ his behavior will be identical to the behavior of a consumer whose entire horizon ends at time $\tau$.  As shown in step I, the MPC always declines with horizon. The MPC for this consumer is therefore strictly greater than the MPC of the unconstrained consumer whose horizon ends at $T > \tau$.  Thus, in each period before $\tau+1$, the consumption function $c_{.,1}$ generated by imposition of the constraint constitutes a counterclockwise concavification of the unconstrained consumption function around the kink point $\wAlt_{.,1}$.
  \end{proof}

  We have now shown the results in Lemma \ref{lem:LcAndCc} for $n = 0$. The last step is to show that they also hold for $n+1$ when they hold strictly for $n$. Consider imposing the $n+1$'st constraint and suppose for concreteness that it applies at the end of period $\tau$. It will stop binding at a level of consumption defined by
  $$u'(c_{\tau,n+1}^{\#}) = \beta R u'(c_{\tau+1,n}({y})) = \beta Ru'(y)$$
  where the second equality follows because a consumer with total resources $y$, constant income, and $\beta R < 1$ will be constrained. But note that by the definition of $c^{\#}_{\tau,n}$, we obtain
  $$	u'(c_{\tau,n}^{\#})  =  (R \beta)^{\mathcal{T}[n]-\tau} u'({y})  <  R \beta u'({y}) = u'(c_{\tau,n+1}^{\#})$$
  where $\mathcal{T}[n]-\tau$ denotes the time remaining to the $n$'th constraint. From the assumption of decreasing marginal utility, we therefore know that
  \begin{eqnarray*}
    c_{\tau,n}^{\#} & \geq & c_{\tau,n+1}^{\#}.
  \end{eqnarray*}
  This means that the constraint is relevant: The pre-existing constraint $n$ does not force the consumer to do so much saving in period $\tau$ that the $n+1$'st constraint fails to bind.

  The prior-period levels of consumption and wealth at which constraint $n+1$ stops impinging on consumption can again be calculated recursively from
  \begin{eqnarray*}
    u'(c_{\tau,n+1}^{\#}) & = & R\beta u'(c_{\tau+1,n}({y}))
    \\  \wAlt_{\tau,n+1} & = & \left(V'_{\tau,n}\right)^{-1}(u'(c_{\tau,n+1}^{\#})).
  \end{eqnarray*}

  Furthermore, once again we can think of the constraint as terminating
  the horizon of a finite-horizon consumer in an earlier period than it
  is terminated for the less-constrained consumer, with the implication
  that the MPC below $\wAlt_{\tau,n+1}$ is strictly greater than the MPC
  above $\wAlt_{\tau,n+1}$.  Thus, the consumption function $c_{\tau,n+1}$
  constitutes a counterclockwise concavification of the consumption
  function $c_{\tau,n}$ around the kink point $\wAlt_{\tau,n+1}$.


  \section{Proof of Theorem \ref{thm:riskandconstraints}}\label{app:riskandconstraints}

  \begin{proof}
                               %%                                Old: We prove Theorem \ref{thm:riskandconstraints} by induction. We first show that it holds when we introduce the first constraint, before we show that it holds when we introduce constraint number $n+1$ when $n$ constraints already exist.

                               %%                                Old: \begin{lemma}(Precautionary Saving with one Liquidity Constraint.) \label{lem:pslc}\\
                  %%                   Old:   Consider an agent who has a utility function with $u'> 0$, $u''< 0$, $u''' > 0$, and non-increasing absolute prudence ($-u'''/u''$). Then
                  %%                   Old:   \begin{equation}
                  %%                   Old:     c_{t,1}({m}) - \tilde{c}_{t,1}({m}) \geq c_{t,0}({m})-\tilde{c}_{t,0}({m}), \label{eq:ineq2}
                  %%                   Old:   \end{equation}
                  %%                   Old:   and the inequality is strict if ${m}_{t} < \bar{\wAlt}_{t,1}$.
                  %%                   Old: \end{lemma}

    Our proof proceeds by constructing the behavior of consumers facing the risk from the behavior of the corresponding perfect foresight consumers.  We consider matters from the perspective of some level of wealth ${{m}}$ for the perfect foresight consumers.  Because the same marginal utility function $u'$ applies to all four consumption rules, the Compensating Precautionary Premia, $\kappa_{t,n}$ and $\kappa_{t,n+1}$, associated with the introduction of the risk $\zeta_{t+1}$  must satisfy
    \begin{eqnarray}
      c_{t,n}({{m}}) & = & \tilde{c}_{t,n}({{m}}+\kappa_{t,n}) \label{eq:hateqtildehat2}
      \\ c_{t,n+1}({{m}}) & = & \tilde{c}_{t,n+1}({{m}}+\kappa_{t,n+1}). \label{eq:hateqtildehat}\end{eqnarray}
    Define the amounts of precautionary saving induced by the risk $\zeta_{t+1}$ at an arbitrary level of wealth ${m}$ in the two cases as
    \begin{eqnarray}
      \psi_{t,n}({m}) & = & c_{t,n}({m})-\tilde{c}_{t,n}({m})
      \\   \psi_{t,n+1}({m}) & = & c_{t,n+1}({m})-\tilde{c}_{t,n+1}({m})
    \end{eqnarray}
    where the mnemonic is that the first two letters of the Greek letter psi stand
    for {\bf p}recautionary {\bf s}aving.

    We can rewrite \eqref{eq:hateqtildehat} (resp. \eqref{eq:hateqtildehat2}) as
    \begin{eqnarray*}
      c_{t,n+1}({m}) = c_{t,n+1}({{m}}+\kappa_{t,n+1})+\int_{{{m}}+\kappa_{t,n+1}}^{{{m}}} c_{t,n+1}^{\prime}(\mu) d\mu  = \tilde{c}_{t,n+1}({{m}}+\kappa_{t,n+1})
      \psi_{t,2}({{m}}+\kappa_{t,2}) \equiv
    \end{eqnarray*}
    which implies that
    \begin{eqnarray*}
      \psi_{t,n+1}({{m}}+\kappa_{t,n+1}) & = &
                                               c_{t,n+1}({{m}}+\kappa_{t,n+1})-\tilde{c}_{t,n+1}({{m}}+\kappa_{t,n+1}) = \int^{{{m}}+\kappa_{t,n+1}}_{{{m}}} c_{t,n+1}^{\prime}(\mu) d\mu,
      \\ %\psi_{t,1}({{m}}+\kappa_{t,1}) \equiv
      \psi_{t,n}({{m}}+\kappa_{t,n}) & = &  c_{t,n}({{m}}+\kappa_{t,n})-\tilde{c}_{t,n}({{m}}+\kappa_{t,n}) = \int^{{{m}}+\kappa_{t,n}}_{{{m}}} c_{t,n}^{\prime}(\mu) d\mu
    \end{eqnarray*}
    and
    \begin{eqnarray}
      \psi_{t,n}({{m}}+\kappa_{t,n+1}) & = &
                                             \psi_{t,n}({{m}}+\kappa_{t,n}) - \int^{{{m}}+\kappa_{t,n+1}}_{{{m}}+\kappa_{t,n}} (\tilde{c}_{t,n}^{\prime}(\mu)-c_{t,n}^{\prime}(\mu)) d\mu \nonumber
    \end{eqnarray}
    so the difference between precautionary saving for the consumer facing $n$ constraints and the one facing $n+1$ constraints at ${{m}}+\kappa_{t,n+1}$ is
    \begin{align}
      \psi_{t,n+1}&({{m}}+\kappa_{t,n+1}) - \psi_{t,n}({{m}}+\kappa_{t,n+1}) = \nonumber\\
                  &= \psi_{t,n+1}({{m}}+\kappa_{t,n+1}) - \psi_{t,n}({{m}}+\kappa_{t,n}) + \psi_{t,n}({{m}}+\kappa_{t,n})- \psi_{t,n}({{m}}+\kappa_{t,n+1}) \nonumber
      \\ &=\int_{{{m}}}^{{{m}}+\kappa_{t,n+1}} c_{t,n+1}^{\prime}(\mu)d\mu - \int^{{{m}}+\kappa_{t,n}}_{{{m}}} c_{t,n}^{\prime}(\mu) d\mu + \int^{{{m}}+\kappa_{t,n+1}}_{{{m}}+\kappa_{t,n}} (\tilde{c}_{t,n}^{\prime}(\mu)-c_{t,n}^{\prime}(\mu)) d\mu \nonumber\\
      % &=\int_{{{m}}}^{{{m}}+\kappa_{t,n}} (c_{t,n+1}^{\prime}(\mu)-c_{t,n}^{\prime}(\mu))d\mu+\int^{{{m}}+\kappa_{t,n+1}}_{{{m}}+\kappa_{t,n}} \left(c_{t,n+1}^{\prime}(\mu)+(\tilde{c}_{t,n}^{\prime}(\mu)-c_{t,n}^{\prime}(\mu)) \right) d\mu \nonumber \\
                  &= \int_{{{m}}}^{{{m}}+\kappa_{t,n+1}}
                    (c_{t,n+1}^{\prime}(\mu)-c_{t,n}^{\prime}(\mu))d\mu +\int^{{{m}}+\kappa_{t,n+1}}_{{{m}}+\kappa_{t,n}}
                    \tilde{c}_{t,n}^{\prime}(\mu) d\mu \label{eq:psidiffint}
    \end{align}


    If we can show that \eqref{eq:psidiffint} is a positive
    number for all feasible levels of ${{m}}$ satisfying ${{m}} < {\bar{\wAlt}}_{t,n+1}$, then we have proven Theorem \ref{thm:riskandconstraints}. We know that the marginal propensity to consume is always strictly positive and that $\kappa_{t,n+1} \geq \kappa_{t,n} \geq 0$\footnote{Since we know that liquidity constraints increase prudence (Corollary \ref{cor:lcip}) and that prudence results in a positive precautionary premium (Lemma \ref{lemma:kimpratt}).} so to prove that \eqref{eq:psidiffint} is strictly positive, we need to show one of two sufficient conditions:
    \begin{enumerate}
    \item $\kappa_{t,n+1} > 0$ and $c_{t,n+1}'(\mu) > c_{t,n}'(\mu)$
    \item $\kappa_{t,n+1} > \kappa_{t,n}$
    \end{enumerate}
    Now, since $u'''>0$, we know that $\kappa_{t,n} > 0$ from Jensen's inequality. Hence, $\kappa_{t,n+1} > 0$ since $\kappa_{t,n+1} \geq \kappa_{t,n}$. The first integral in \eqref{eq:psidiffint} is therefore strictly positive as long as $c_{t,n+1}' > c_{t,n}'$, which is true for ${{m}} < \wAlt_{t,n+1}$ by Lemma \ref{lem:LcAndCc}.

    For ${{m}} \geq \wAlt_{t,n+1}$, we know that $c_{t,n+1}' = c_{t,n}'$ so the first integral in \eqref{eq:psidiffint} is always zero. For the second integral in \eqref{eq:psidiffint} to be strictly positive, we need to show that $\kappa_{t,n+1} > \kappa_{t,n}$.

                    %%                     Old: We will prove this by using a proof by showing that assuming $\kappa_{t,n+1} \leq \kappa_{t,n}$ leads to the conclusion that $\kappa_{t,n+1} > \kappa_{t,n}$.

                    %                     First define $\hat{\zeta}$ as the realization of $\zeta$ such that
                    %                     $$ {a}_{t,n+1} + y + \kappa_{t,n+1} + \zeta = \omega_{t,n+1}.$$
                    %                     In words, $\hat{\zeta}$ is the value of $\zeta$ that cancels the values of ${a}_{t,n+1}$ and $\kappa_{t,n+1}$ relative to the $n+1$'st constraint in the sense that given the level of savings and precautionary premia, a draw if $\zeta = \hat{\zeta}$ restores wealth to exactly the value at which constraint $n+1$ stops binding.
                    %                     XXXXX



    First define the perfect foresight consumption functions as
    \begin{eqnarray}
      c(\kappa_{t,n}+\zeta) & = & c_{t+1,n}(\overbrace{{s}_{t,n}}^{={a}_{t,n+1}}+{y}+\kappa_{t,n}+\zeta) \label{eq:cnoconstr}
      \\  {c}(\kappa_{t,n+1}+\zeta) & = & c_{t+1,n+1}({s}_{t,n+1}+{y}+\kappa_{t,n+1}+\zeta)\label{eq:gravecnoconstr}.
    \end{eqnarray}	
    where ${a}_{t,n} = {a}_{t,n+1}$ since ${m} \geq \wAlt_{t,n+1}$. Recall also the definitions of $\kappa_{t,n}$ and $\kappa_{t,n+1}$:
    \begin{eqnarray*}
      u'({c}_{t,n}) & = & \Ex_{t}[u'(c(\kappa_{t,n}+\zeta))]
      \\  u'({c}_{t,n+1}) & = & \Ex_{t}[u'({c}(\kappa_{t,n+1}+\zeta))].
    \end{eqnarray*}

    Now recall that Lemma~\ref{lemma:kimpratt} tells us that if absolute prudence of $u'(c(\kappa_{t,n}+\zeta))$ is identical to absolute prudence of $u'({c}(\kappa_{t,n+1}+\zeta))$ for every realization of $\zeta$, then $\kappa_{t,n}=\kappa_{t,n+1}$. This is true if ${m}_{t+1} \geq \wAlt_{t+1,n+1}$ for all possible realizations of $\zeta \in (\underline{\zeta}, \bar{\zeta})$, i.e. that the agent is unconstrained for all realizations of the risk. We defined this limit as ${m}_{t+1} \geq {\bar{\wAlt}}_{t+1,n+1}$. We therefore know that $\kappa_{t,n+1}  = \kappa_{t,n}$ if ${{m}} \geq {\bar{\wAlt}}_{t+1,n+1}$.

    For all levels of wealth below this limit (${{m}} < {\bar{\wAlt}}_{t+1,n+1}$), there exist realizations of $\zeta$ such that constraint $n+1$ will bind in period $t+1$. The agent will require a higher precautionary premia when facing constraint $n+1$ in addition to the $n$ constraints already in the set, implying that $\kappa_{t,n+1} > \kappa_{t,n}$. Equation \eqref{eq:psidiffint} is therefore strictly positive if ${{m}} < {\bar{\wAlt}}_{t+1,n+1}$ and we have proven Theorem \ref{thm:riskandconstraints}.
                                %%                                 Old: \subsection*{The $n+1$'st constraint}
                                %%                                 Old: Consider now the case where we have imposed $n$ constraints and are considering imposing constraint $n+1$ and where constraint $n+1$ applies at the end of some future period. Similar to the introduction of the first constraint, we need to show that the following equation is strictly positive:
                                %%                                 Old: \begin{align}
                                %%                                 Old:   \psi_{t,n+1}&({{m}}+\kappa_{t,n+1}) - \psi_{t,n}({{m}}+\kappa_{t,n+1}) = \nonumber\\ &=\int_{{{m}}}^{{{m}}+\kappa_{t,n}} (c_{t,n+1}^{\prime}(\mu)-c_{t,n}^{\prime}(\mu))d\mu+\int^{{{m}}+\kappa_{t,n+1}}_{{{m}}+\kappa_{t,n}} \left(c_{t,n+1}^{\prime}(\mu)+(\tilde{c}_{t,n}^{\prime}(\mu)-c_{t,n}^{\prime}(\mu)) \right) d\mu \nonumber \\
                  %%                   Old:   &= \int_{{{m}}}^{{{m}}+\kappa_{t,n+1}}
                                                %%                                                 Old:   (c_{t,n+1}^{\prime}(\mu)-c_{t,n}^{\prime}(\mu))d\mu +\int^{{{m}}+\kappa_{t,n+1}}_{{{m}}+\kappa_{t,n}}
                                                %%                                                 Old:   \tilde{c}_{t,n}^{\prime}(\mu) d\mu \label{eq:psidiffint2}
                                                %%                                                 Old: \end{align}
                                                %%                                                 Old: The sufficient conditions for \eqref{eq:psidiffint2} to be strictly positive are
                                                %%                                                 Old: \begin{enumerate}
                                                %%                                                 Old:  \item $\kappa_{t,n+1} > 0$ or $\kappa_{t,n} > 0$, and $c_{t,n+1}'(\mu) > c_{t,n}'(\mu)$
                                                %%                                                 Old:  \item $\kappa_{t,n+1} > \kappa_{t,n}$
                                                %%                                                 Old: \end{enumerate}
                                                %%                                                 Old:  Now since $u'''>0$, we know that $\kappa_{t,n} > 0$ from Jensen's inequality and Lemma \ref{lemma:kimpratt}. The first integral in \eqref{eq:psidiffint2} is therefore strictly positive as long as $c_{t,n+1}' > c_{t,n}'$, which is true for ${{m}} < \wAlt_{t,n+1}$ (Theorem \ref{thm:pfclc}).
                                                %%                                                 Old:
                                                %%                                                 Old:  For ${{m}} \geq \wAlt_{t,n+1}$, we know that $c_{t,n+1}' = c_{t,n}'$ so the first integral in \eqref{eq:psidiffint2} is always zero. For the second integral in \eqref{eq:psidiffint2} to be strictly positive, we need to show that $\kappa_{t,n+1} > \kappa_{t,n}$. Recall the definition of $\kappa_{t,n}$ and $\kappa_{t,n+1}$:
                                                %%                                                 Old:  \begin{eqnarray*}
                                                %%                                                 Old:   u'({c}_{t,n}) & = & \Ex_{t}[u'(c(\kappa_{t,n}+\zeta))]
                                                                                                                              %%                                                                                                                               Old:   \\  u'({c}_{t,n+1}) & = & \Ex_{t}[u'(\hat{c}(\kappa_{t,n+1}+\zeta))],
                                                                                                                                                                                                                                                                                                %%                                                                                                                                                                                                                                                                                                 Old:  \end{eqnarray*}
                                                                                                                                                                                                                                                                                                %%                                                                                                                                                                                                                                                                                                 Old:  where the two consumption functions are defined as
                                                                                                                                                                                                                                                                                                %%                                                                                                                                                                                                                                                                                                 Old:  \begin{eqnarray}
                                                                                                                                                                                                                                                                                                %%                                                                                                                                                                                                                                                                                                 Old:   c(\kappa_{t,n}+\zeta) & = & c_{t+1,n}(\overbrace{{s}_{t,n}}^{={s}_{t,n+1}}+{y}+\kappa_{t,n}+\zeta) \label{eq:cnoconstr2}
                                                                                                                                                                                                                                                                                                                                                                                                                                                                                                                                                                                                                                      %%                                                                                                                                                                                                                                                                                                                                                                                                                                                                                                                                                                                                                                       Old:   \\  \hat{c}(\kappa_{t,n+1}+\zeta) & = & c_{t+1,n+1}({s}_{t,n+1}+{y}+\kappa_{t,n+1}+\zeta)\label{eq:gravecnoconstr2}.
                                                                                                                                                                                                                                                                                                                                                                                                                                                                                                                                                                                                                                                                                                                                                                                                                                                                                                                                                                                                                                                                                                                                                                                                                                                                                                                              %%                                                                                                                                                                                                                                                                                                                                                                                                                                                                                                                                                                                                                                                                                                                                                                                                                                                                                                                                                                                                                                                                                                                                                                                                                                                                                                                               Old:  \end{eqnarray}
                                                                                                                                                                                                                                                                                                                                                                                                                                                                                                                                                                                                                                                                                                                                                                                                                                                                                                                                                                                                                                                                                                                                                                                                                                                                                                                              %%                                                                                                                                                                                                                                                                                                                                                                                                                                                                                                                                                                                                                                                                                                                                                                                                                                                                                                                                                                                                                                                                                                                                                                                                                                                                                                                               Old:	Now recall that Lemma~\ref{lemma:kimpratt} tells us that if absolute prudence of $u'(c(\kappa_{t,n}+\zeta))$ is identical to absolute prudence of $u'(\hat{c}(\kappa_{t,n+1}+\zeta))$ for every realization of $\zeta$, then $\kappa_{t,n}=\kappa_{t,n+1}$. This can only be true if ${m}_{t+1} \geq \wAlt_{t+1,n+1}$ for any realization of $\zeta \in (\underline{\zeta}, \bar{\zeta})$. We defined this limit as ${m}_{t+1} \geq {\bar{\wAlt}}_{t,n+1}$. We therefore know that $\kappa_{t,n+1}  = \kappa_{t,n}$ if ${{m}} \geq {\bar{\wAlt}}_{t,n+1}$ and $\kappa_{t,n+1} > \kappa_{t,n}$ if ${{m}} < {\bar{\wAlt}}_{t,n+1}$. Equation \eqref{eq:psidiffint2} is therefore strictly positive if ${{m}} < {\bar{\wAlt}}_{t,n+1}$ and we have proven Theorem \ref{thm:riskandconstraints}.
                                                                                                                                                                                                                                                                                                                                                                                                                                                                                                                                                                                                                                                                                                                                                                                                                                                                                                                                                                                                                                                                                                                                                                                                                                                                                                                              %%
  \end{proof}

  \section{Resemblance Between Precautionary Saving and a Liquidity Constraint} \label{app:similar}

  This appendix repeats an illustration from appendix G of \cite{BufferStockTheory}.  (We make no claim to novelty of this point; it is here only to aid the intuition of the reader).

  In this appendix, we provide an example where the introduction of risk resembles the introduction of a constraint. Consider the second-to-last period of life for two risk-averse CRRA utility consumers and assume for simplicity that $R=\beta=1$.

  The first consumer is subject to a liquidity constraint $c_{T-1} \geq {m}_{T-1}$, and earns non-stochastic income of ${y}=1$ in period $T$. This consumer's saving rule will be
  \begin{equation*}
    {a}_{T-1,1}({m}_{T-1}) =
    \begin{cases}
      0   & \mbox{if ${m}_{T-1} \leq 1$}
      \\    ({m}_{T-1}-1)/2 & \mbox{if ${m}_{T-1} > 1$}.
    \end{cases}
  \end{equation*}

  The second consumer is not subject to a liquidity constraint, but
  faces a stochastic income process,
  \begin{equation*}
    y_{T} =
    \begin{cases}
      0 & \mbox{with probability $p$}
      \\ \frac{1}{1-p} & \mbox{with probability $(1-p)$}.
    \end{cases}
  \end{equation*}

  If we write the consumption rule for the unconstrained consumer facing the risk as $\tilde{s}_{T-1,0},$ the key result is that in the limit as $p \downarrow 0$, behavior of the two consumers becomes the same. That is, defining $\tilde{s}_{T-1,0}({m})$ as the optimal saving rule for the consumer facing the risk,
  \begin{equation*}
    \lim_{p \downarrow 0} \tilde{s}_{T-1,0}({m}_{T-1}) = {a}_{T-1,1}({m}_{T-1}) \label{eq:lim0}
  \end{equation*}
  for every ${m}_{T-1}$.

  To see this, start with the Euler equations for the two consumers given wealth ${m}$,
  \begin{eqnarray}
    u'({m}-{a}_{T-1,1}({m})) & = & u'({a}_{T-1,1}({m})+1) \label{eq:euler1}
    \\   u'({m}-\tilde{s}_{T-1,0}({m})) & = & p u'(\tilde{s}_{T-1,0}({m}))+(1-p) u'(\tilde{s}_{T-1,0}({m})+1). \label{eq:euler2}
  \end{eqnarray}

  Consider first the case where ${m}$ is large enough that the constraint does not bind for the constrained consumer, ${m}>1$.  In this case the limit of the Euler equation for the second consumer is identical to the Euler equation for the first consumer (because for ${m}>1$ savings are positive for the consumer facing the risk, implying that the limit of the first $u'$ term on the RHS of \eqref{eq:euler2} is finite). Thus the limit of \eqref{eq:euler2} is \eqref{eq:euler1} for ${m}>1$.

  Now consider the case where ${m} < 1$ so that the first consumer would be constrained.  This consumer spends her entire resources ${m}$, and by the definition of the constraint we know that
  \begin{eqnarray}
    u'({m}) > u'(1). \label{eq:upw}
  \end{eqnarray}

  Now consider the consumer facing the risk. If this consumer were to save exactly zero and then experienced the bad shock in period $T$, she would have an infinite marginal utility (the Inada condition). This cannot satisfy the Euler-equation as long as ${m}>0$. Therefore we know that for any $p>0$ and any ${m}>0$ the consumer will save some positive amount. For a fixed ${m}$, hypothesize that there is some $\delta>0$ such that no matter how small $p$ became the consumer would always choose to save at least $\delta$. But for any $\delta$, the limit of the RHS of \eqref{eq:euler2} is $u'(1+\delta)$. We know from concavity of the utility function that $u'(1+\delta) < u'(1)$ and we know from~\eqref{eq:upw} that $u'({m}) > u'(1) > u'(1+\delta)$, so as $p \downarrow 0$ there must always come a point at which the consumer can improve her total utility by shifting some resources from the future to the present, i.e. by saving less.  Since this argument holds for any $\delta>0$ it demonstrates that as $p$ goes to zero there is no positive level of saving that would make the consumer better off. But saving of zero or a negative amount is ruled out by the Inada condition at $u'(0)$.  Hence saving must approach, but never equal, zero as $p \downarrow 0$.

  Thus, we have shown that for ${m} \leq 1 $ and for ${m} > 1$ in the limit as $p \downarrow 0$ the consumer facing the risk but no constraint behaves identically to the consumer facing the constraint but no risk. This argument can be generalized to show that for the CRRA utility consumer, spending must always be strictly less than the sum of current wealth and the minimum possible value of human wealth.  Thus, the addition of a risk to the problem can rule out certain levels of wealth as feasible, and can also render either future or past constraints irrelevant, just as the imposition of a new constraint can.


  \section{Proof of Theorem \ref{thm:CCandPS}}\label{app:CCandPS}

  \begin{proof}
    To simplify notation and without loss of generality, we assume that when an agent faces $n$ constraints and $m$ risks, there are one constraint and one risk for each time period. For example, if $c_{t,n}^{m}$ faces $m$ future risks and $n$ future constraints, then the next period consumption function is $c_{t+1,n-1}^{m-1}$ (and $m = n$). Note that we can transform any problem into this notation by filling in with degenerate risks and non-binding constraints. However, for Theorem \ref{thm:CCandPS} to hold with strict inequality, we need to assume that there is at least one relevant future risk and one relevant constraint.

    We know that either the introduction of risk or a introduction of a constraint results in a counterclockwise concavification of the original consumption function. However, this is only true when we introduce risks in the absence of constraints (Lemma \ref{lem:ccrisk}) and when we introduce constraints in the absence of risk (see Theorem \ref{thm:lcip2}). In this proof, we therefore need to show that the introduction of all risks and constraints is a counterclockwise concavification of the linear case with no risks and constraints.

    Here is our proof strategy. We define a set
    \begin{align}
      \mathcal{P}_{t,n}^{m-1} = \{ {m} | c_{t,n}^{m-1}({m})-c_{t,n}^{m}({m})>c_{t,0}({m})-c^{1}_{t,0}({m}) \} \label{eq:whatcanbesaid_proof}
    \end{align}
    where Theorem \ref{thm:CCandPS} holds in period $t$ when we introduce a risk at the beginning of period $t+1$. This is defined as the set where precautionary saving induced by a risk that is realized at the beginning of period $t+1$ is greater in the presence of all risks and constraints than in the unconstrained case.

    In order to show that the set $\mathcal{P}_{t,n}^{m-1}$ is non-empty, we build it up recursively, starting from period $T$ and adding one constraint or one risk  for each time period. The key to the proof is to understand that the introduction of risks or constraints will never fully reverse the effects of all other risks and constraints, even though they sometimes reduce absolute prudence for some levels of wealth because risks and constraints can mask the effects of future risks and constraints. Hence, the new consumption function must still be a counterclockwise concavification of the consumption function with no risks and constraints for some levels of wealth.

    Since a counterclockwise concavification increases prudence by Lemma \ref{lem:CCToPrud}, and higher prudence increases precautionary saving by Lemma \ref{lemma:kimpratt}, our required set can be redefined as
    \begin{align*} \mathcal{P}_{t,n}^{m-1} = \{{m} | c_{t,n}^{m-1}({m}) \text{ is a counterclockwise concavification of } c_{t,0}({m}) \text{ and } c_{t,n}^{m-1}({m}) >
      {m} 	\}\end{align*}
    where we add the last condition, $c_{t,n}^{m-1}({m}) > w$ to avoid the possibility that some constraint binds such that the agent does not increase precautionary saving. In words: $\mathcal{P}_{t,n}^{m-1}$ is the set where the consumption function is a counterclockwise concavification of $c_{t,0}({m})$ and no constraint is strictly binding. We construct the set recursively for two different cases: CARA and all other type of utility functions. We start with the non-CARA utility functions.

    First add the last constraint. The set $\mathcal{P}_{T,1}^0$ is then
    \[\mathcal{P}_{T,1}^0 = \emptyset \]%\left\{ w| {m} < \omega_{T,1}  \right\}\]
    since we know that $c_{T,1}({m})$ is a counterclockwise concavification of $c_{T,0}({m})$ around $\wAlt_{T,1}$ but that the consumer is constrained below this point.

    We next add the risk at the beginning of period $T$. To construct the new set, we note three things. First, by Lemma \ref{lem:recursive},  (strict) consumption concavity is recursively propagated for all values of wealth where there is a positive probability that the constraint can bind, i.e.
    \[\{{m}_{T-1} | \wAlt_{T-1,1} \in \left[{m}_{T-1} - c_{T-1,1}^1({m}_{T-1}) + {y}_{T} + \underline{\zeta} ,{m}_{T-1} - c_{T-1,1}^1({m}_{T-1}) + {y}_{T} + \bar{\zeta}\right] \}\]
    has property strict CC, while it has non-strict property $CC$ for all possible values of ${m}_{T-1}$. Further, we know from Theorem \ref{thm:riskandconstraints} (rearrange equation \eqref{eq:ineq}) that
    \begin{align*}{c}_{T-1,1}^1({m})  & \leq c_{T-1,1}({m}) - c_{T-1,0}({m}) + {c}_{T-1,0}^1({m}) \\ &\leq c_{T-1,0}({m}) - c_{T-1,0}({m}) + c_{T-1,0}({m}) = c_{T-1,0}({m}).\end{align*}
    Third, we know that $c_{T-1,1}^{1}({m})' \geq c_{T-1,0}'({m})$ since ${c}_{T-1,1}^1({m}) < c_{T-1,0}({m})$ for ${m} \leq \wAlt_{T-1,1}^1$, $\lim_{{m}\rightarrow \infty} c_{T-1,1}^1({m}) - c_{T-1,0}({m}) = 0$, and that $c_{t,1}^1({m})$ is concave while $c_{t,0}({m})$ is linear. Hence, $c_{T-1,1}^1$ is a counterclockwise concavification of $c_{T-1,0}$ around the minimum value of wealth when the constraint will never bind and the new set is
    \begin{align*}\mathcal{P}_{T-1,1}^1 = \{{m}_{T-1} | \wAlt_{T-1,1} \in \left[{m}_{T-1} - c_{T-1,1}^1({m}_{T-1}) + {y}_{T} + \underline{\zeta}, {m}_{T-1} - c_{T-1,1}^1({m}_{T-1}) + {y}_{T} + \bar{\zeta}\right]   \\ \wedge c_{T-1,1}^1({m}) > {m}_{T-1}\}.\end{align*}

    We can now add the next constraint. The consumption function now has two kink points, $\wAlt_{T-1,1}^1$ and $\wAlt_{T-1,2}^1$. We know again from Lemma \ref{lem:recursive} that consumption concavity is preserved when we add a constraint, and strict consumption concavity is preserved for all values of wealth at which a future constraint might bind. Further, we know from Theorem \ref{thm:riskandconstraints} that
    \begin{align*}c_{T-1,2}^1({m}) &\leq c_{T-1,2}({m}) - c_{T-1,1}({m}) + c_{T-1,1}^1({m}) \\ &\leq c_{T-1,1}({m}) - c_{T-1,1}({m}) + c_{T-1,1}({m}) = c_{T-1,1}({m}) \leq c_{T-1,0}({m}). \end{align*}
    Third, $c_{T-1,2}^1({m}) < c_{T-1,0}({m})$,  $\lim_{{m}\rightarrow\infty} c_{T-1,2}^1({m}) - c_{T-1,0}({m}) = 0$, and we know that if $c_{T-1,2}^1({m})$ is concave while $c_{T-1,0}({m})$ is linear, then $c_{T-1,2}'^{1}({m}) \geq c_{T-1,0}'({m})$. $c_{T-1,2}^1({m})$ which is a counterclockwise concavification of $c_{T-1,0}({m})$ around the minimum level of wealth at which the first constraint will never impinge on time $T-1$ consumption, $\bar{\wAlt}_{T-1,1}^1$, and the new set is
    \[\mathcal{P}_{T-1,2}^1 = \{{m}_{T-1}| {m}_{T-1} \leq \bar{\wAlt}_{T-1,1}^1 \wedge c_{T-1,2}^1({m}) > {m}_{T-1} \}.\]

    It is now time to add the next risk. The argument is similar. We still know that (strict) consumption concavity is recursively propagated and that $\lim_{{m}\rightarrow \infty} c_{T-2,2}^2({m}) - c_{T-2,0}({m}) = 0$. Further, we can think of the addition of two risks over two periods as adding one risk that is realized over two periods. Hence, the results from Theorem \ref{thm:riskandconstraints} must hold also for the addition of multiple risks so we have
    \begin{align*}c_{T-2,2}^2({m}) &\leq c_{T-2,2}({m}) - c_{T-2,1}({m}) + c_{T-2,1}^2({m}) \\ &\leq c_{T-2,1}({m}) - c_{T-2,1}({m}) + c_{T-2,1}({m}) = c_{T-2,1}({m}) \leq c_{T-2,0}({m}). \end{align*}
    Hence, we again know that $c_{T-2,2}'^{2}({m}) \geq c_{T-2,0}'({m})$. $c_{T-2,2}^2({m})$ is thus a counterclockwise concavification of $c_{T-2,0}({m})$ around the level of wealth at minimum value of wealth when the last constraint will never bind. The new set is therefore
    \[\mathcal{P}_{T-2,2}^2 = \{{m}_{T-2} | {m}_{T-2} - c_{T-2,2}^2({m}_{T-2}) + {y}_{T-1} + \zeta_{T-1} \in \mathcal{P}_{T-1,2}^1 \wedge c_{T-2,2}^{2}({m}) < {m} \}.\]


    Doing this recursively and defining $\bar{\wAlt}_{t,1}^{m-1}$ as the minimum value of wealth beyond which constraint $1$ will never bind, the set of wealth levels at which Theorem \ref{thm:CCandPS} holds can be defined as
    \[\mathcal{P}_{t,n}^{m-1} = \{{m}_{t} | {m}_{t} \leq \bar{\wAlt}_{t,1}^{m-1} \wedge c_{t,n}^{m-1}({m}) > {m} \}\]
    In words, precautionary saving is higher if there is a positive probability that some future constraint could bind and the consumer is not constrained today.

    The last requirement is to define the set also for the CARA utility function. The problem with CARA utility is that $\lim_{{m}\rightarrow \infty} c_{t,n}^{m-1}({m}) - c_{t,0}({m}) = - k^{m-1} \leq 0$ where $k^{m-1}$ is some positive constant. We can therefore not use the same arguments as in the preceding proof. However, by realizing that equation \eqref{eq:ineq} in the CARA case can be defined as
    \begin{align*}
      c_{t,n+1}({m}) - \tilde{c}_{t,n+1}({m}) - \tilde{k} &\geq c_{t,n}({m})-\tilde{c}_{t,n}({m}) -  \tilde{k} \geq 0
    \end{align*}
    where the last inequality follows since precautionary saving is always higher than in the constant limit in the presence of constraints. We can therefore rearrange to get
    \begin{align*}
      \tilde{c}_{t,n+1}({m})  &\leq c_{t,n+1}({m}) - \tilde{k} \leq c_{t,n}({m}) - \tilde{k} \leq c_{t,0} - \tilde{k}	\end{align*}
    which implies that the arguments in the preceding section goes through also for CARA utility with this slight modification.

  \end{proof}

